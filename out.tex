% Options for packages loaded elsewhere
\PassOptionsToPackage{unicode}{hyperref}
\PassOptionsToPackage{hyphens}{url}
%
\documentclass[
]{article}

\let\oldsection\section
\renewcommand\section{\clearpage\oldsection}


\usepackage{amsmath,amssymb}
\usepackage{iftex}
\ifPDFTeX
  \usepackage[T1]{fontenc}
  \usepackage[utf8]{inputenc}
  \usepackage{textcomp} % provide euro and other symbols
\else % if luatex or xetex
  \usepackage{unicode-math} % this also loads fontspec
  \defaultfontfeatures{Scale=MatchLowercase}
  \defaultfontfeatures[\rmfamily]{Ligatures=TeX,Scale=1}
\fi
\usepackage{lmodern}
\ifPDFTeX\else
  % xetex/luatex font selection
\fi
% Use upquote if available, for straight quotes in verbatim environments
\IfFileExists{upquote.sty}{\usepackage{upquote}}{}
\IfFileExists{microtype.sty}{% use microtype if available
  \usepackage[]{microtype}
  \UseMicrotypeSet[protrusion]{basicmath} % disable protrusion for tt fonts
}{}
\makeatletter
\@ifundefined{KOMAClassName}{% if non-KOMA class
  \IfFileExists{parskip.sty}{%
    \usepackage{parskip}
  }{% else
    \setlength{\parindent}{0pt}
    \setlength{\parskip}{6pt plus 2pt minus 1pt}}
}{% if KOMA class
  \KOMAoptions{parskip=half}}
\makeatother
\usepackage[margin=0.5in]{geometry}
\usepackage{xcolor}
\setlength{\emergencystretch}{3em} % prevent overfull lines
\providecommand{\tightlist}{%
  \setlength{\itemsep}{0pt}\setlength{\parskip}{0pt}}
\setcounter{secnumdepth}{5}
\ifLuaTeX
  \usepackage{selnolig}  % disable illegal ligatures
\fi
\IfFileExists{bookmark.sty}{\usepackage{bookmark}}{\usepackage{hyperref}}
\IfFileExists{xurl.sty}{\usepackage{xurl}}{} % add URL line breaks if available
\urlstyle{same}
\hypersetup{
  hidelinks,
  pdfcreator={LaTeX via pandoc}}

\author{}
\date{}

\begin{document}

\section{Astrodynamics and Space Applications Qualifying Exam
Notes}\label{astrodynamics-and-space-applications-qualifying-exam-notes}

gps

orb

quals

sad

stm

\section{Satellite Navigation Past
Problems}\label{satellite-navigation-past-problems}

\section{Orbit Mechanics Past
Problems}\label{orbit-mechanics-past-problems}

\subsection{Spring 2021}\label{spring-2021}

\subsubsection{Problem Statement}\label{problem-statement}

Assume a system of four centrobaric bodies that can all move in any
spatial dimension.

\begin{enumerate}
\tightlist
\item
  From first principles, derive the vector differential equation
  governing relative motion. It is not possible to solve the
  corresponding scalar equations of motion. Why not?
\item
  Derive expressions for the 10 known integrals of motion associated
  with this vector differential equation for the 4-body system. What is
  the physical significance of each?
\item
  The motion of the Moon relative to the Earth, and influenced by the
  Sun, is one of the most challenging problems in orbital mechanics.
  Given the results in (a) and (b), discuss why the first statement is
  true.
\end{enumerate}

\subsubsection{Solution}\label{solution}

The vector from the \(i\)th body to the \(j\)th body is given by:

\[\begin{aligned}
\begin{aligned}
    \mathbf{r}_{ij} &= \mathbf{r}_j - \mathbf{r}_i \\
\end{aligned}
\end{aligned}\]

Taking two derivatives with respect to time:

\[\begin{aligned}
\begin{aligned}
    \ddot{\mathbf{r}}_{ij} &= \ddot{\mathbf{r}}_j - \ddot{\mathbf{r}}_i \\
\end{aligned}
\end{aligned}\]

The acceleration of the \(i\)th body is given by:

\[\begin{aligned}
\begin{aligned}
    \ddot{\mathbf{r}}_i &= - \sum_{k\neq i} \frac{G m_k}{\left| \mathbf{r}_{ki} \right|^3} \mathbf{r}_{ki} \\
\end{aligned}
\end{aligned}\]

The acceleration of the \(j\)th body is given by:

\[\begin{aligned}
\begin{aligned}
    \ddot{\mathbf{r}}_j &= -\sum_{k\neq j} \frac{G m_k}{\left| \mathbf{r}_{kj} \right|^3} \mathbf{r}_{kj} \\
\end{aligned}
\end{aligned}\]

Such that the relative acceleration of the \(i\)th body with respect to
the \(j\)th body is given by:

\[\begin{aligned}
\begin{aligned}
    \ddot{\mathbf{r}}_{ij} &= -\sum_{k\neq j} \frac{G m_k}{\left| \mathbf{r}_{kj} \right|^3} \mathbf{r}_{kj} + \sum_{k\neq i} \frac{G m_k}{\left| \mathbf{r}_{ki} \right|^3} \mathbf{r}_{ki} \\
\end{aligned}
\end{aligned}\]

We can pull out the \(k=i\) in the first sum and \(k=j\) in the second
sum:

\[\begin{aligned}
\begin{aligned}
    \ddot{\mathbf{r}}_{ij} &= -\frac{G m_i}{\left| \mathbf{r}_{ji} \right|^3} \mathbf{r}_{ji} + \frac{G m_j}{\left| \mathbf{r}_{ij} \right|^3} \mathbf{r}_{ij} -\sum_{k\neq j, k\neq i} \frac{G m_k}{\left| \mathbf{r}_{kj} \right|^3} \mathbf{r}_{kj} + \sum_{k\neq i, k\neq j} \frac{G m_k}{\left| \mathbf{r}_{ki} \right|^3} \mathbf{r}_{ki} \\
\end{aligned}
\end{aligned}\]

Combining the first two terms and the last two terms:

\[\begin{aligned}
    \ddot{\mathbf{r}}_{ij} &= -\frac{G (m_i + m_j)}{\left| \mathbf{r}_{ji} \right|^3} \mathbf{r}_{ji} - \sum_{k\neq j, k\neq i} G m_k \left(\frac{\mathbf{r}_{kj}}{{\left| \mathbf{r}_{kj} \right|^3}} - \frac{\mathbf{r}_{ki}}{{\left| \mathbf{r}_{ki} \right|^3}}\right)
\end{aligned}\]

Which in the case of the 4-body problem, we take the \(i=1\) and
\(j=2\), \(k \in [3,4]\):

\[\begin{aligned}
    \ddot{\mathbf{r}}_{12} &= -\frac{G (m_1 + m_2)}{\left| \mathbf{r}_{21} \right|^3} \mathbf{r}_{21} - \sum_{k=3}^4 G m_k \left(\frac{\mathbf{r}_{k2}}{{\left| \mathbf{r}_{k2} \right|^3}} - \frac{\mathbf{r}_{k1}}{{\left| \mathbf{r}_{k1} \right|^3}}\right)
\end{aligned}\]

Deriving the 10 known integrals of motion begins by first noting that
the sum of all forces on the system is zero:

\[\sum_i F = \sum_i \sum_{j \neq i} \left( -\frac{G m_i m_j}{r_{ji}^3} \mathbf{r}_{ji} \right)\]

Because
\(F_{sys} = m_{sys} a_{sys} = \sum_i m_i \ddot{\mathbf{r}}_i = 0\) for
the system as a whole, we can state:

\[\int \sum_i m_i \ddot{\mathbf{r}}_i \: dt = m_{sys} v_{sys} = \sum_i m_i \dot{\mathbf{r}}_i\]

Integrating once more:

\[\int \sum_i m_i \dot{\mathbf{r}}_i \: dt = m_{sys} v_{sys} t + m_{sys} r_{sys}  = \sum_i m_i \mathbf{r}_i\]

These two constants of integration \(\mathbf{r}_{sys}\) and
\(\mathbf{v}_{sys}\) are the first two integrals of motion (making up
six scalar equations). Specifically, they are solved for by dividing by
\(m_{sys}\):

\[\begin{aligned}
\begin{aligned}
    \mathbf{r}_{sys} &= \frac{1}{m_{sys}} \sum_i m_i \mathbf{r}_i \\
    \mathbf{v}_{sys} &= \frac{1}{m_{sys}} \sum_i m_i \dot{\mathbf{r}}_i \\
\end{aligned}
\end{aligned}\]

The next three integrals of motion are found by taking the summation of
the angular momentum of the system. We must develop this by taking the
sum of the torques on members of the system. First, we note that the
torque on the \(i\)th body is given by:

\[\begin{aligned}
\begin{aligned}
    \mathbf{\tau}_i &= \mathbf{r}_i \times \mathbf{F}_i \\
    &= \mathbf{r}_i \times \sum_{j \neq i} \left( -\frac{G m_i m_j}{r_{ji}^3} \mathbf{r}_{ji} \right) \\
\end{aligned}
\end{aligned}\]

Such that the total torque on the system is:

\[\begin{aligned}
\begin{aligned}
    \sum_i \mathbf{\tau}_i &= \sum_i \mathbf{r}_i \times \sum_{j \neq i} \left( -\frac{G m_i m_j}{r_{ji}^3} \mathbf{r}_{ji} \right) \\
    &= \sum_i \sum_{j \neq i} \mathbf{r}_i \times \left( -\frac{G m_i m_j}{r_{ji}^3} \mathbf{r}_{ji} \right) \\
\end{aligned}
\end{aligned}\]

We then note that
\(\mathbf{r}_i \times \mathbf{r}_{ji} = -\mathbf{r}_{ji} \times \mathbf{r}_i\),
such that each term in the summation is annihilated by its counterpart.
This leaves us with:

\[\begin{aligned}
\begin{aligned}
    \sum_i \mathbf{\tau}_i &= \mathbf{0} \\
    &= \sum_i \mathbf{r}_i \times m_i \ddot{\mathbf{r}}_i
\end{aligned}
\end{aligned}\]

We notice that this summation expansion of the torque is the derivative
of another quantity:

\[\begin{aligned}
\begin{aligned}
    \sum_i \mathbf{r}_i \times m_i \ddot{\mathbf{r}}_i &= \frac{d}{dt} \left( \sum_i \mathbf{r}_i \times m_i \dot{\mathbf{r}}_i \right) \\
\end{aligned}
\end{aligned}\]

Which implies that the integral of the derived quantity is an integral
of motion:

\[\begin{aligned}
\begin{aligned}
    \sum_i \mathbf{r}_i \times m_i \dot{\mathbf{r}}_i &= \mathbf{h}_{sys} \\
\end{aligned}
\end{aligned}\]

Finally, we enforce conservation of energy by first finding the
potential function whose gradient is the force on the system:

\[\begin{aligned}
\begin{aligned}
    U &= \frac{1}{2} G \sum_i \sum_{j \neq i} \frac{m_i m_j}{r_{ji}} \\
\end{aligned}
\end{aligned}\]

Such that the gradient of the potential function is:

\[\begin{aligned}
\begin{aligned}
    \nabla U &= -G \sum_i \sum_{j \neq i} \frac{m_i m_j}{r_{ji}^3} \mathbf{r}_{ji} \\
    &= \sum_i m_i \ddot{\mathbf{r}} \\
\end{aligned}
\end{aligned}\]

Notice that we can express:

\[\begin{aligned}
\begin{aligned}
    \sum_i \nabla U &= \sum_i\frac{d U}{d \mathbf{r}_i} \\
    &= \sum_i m_i \ddot{\mathbf{r}}
\end{aligned}
\end{aligned}\]

If we multiply both sides by \(\dot{\mathbf{r}}\):

\[\begin{aligned}
\begin{aligned}
    \sum_i m_i \dot{\mathbf{r}}_i \cdot \ddot{\mathbf{r}}_i &= \sum_i\frac{d U}{d \mathbf{r}_i} \frac{d \mathbf{r}_i}{dt} \\
    &= \frac{d U}{dt} \\
\end{aligned}
\end{aligned}\]

We then notice that the left side can also be expressed as the
derivative of a quantity:

\[\begin{aligned}
\begin{aligned}
    \frac{dU}{dt} &= \frac{d}{dt} \left( \sum_i m_i \dot{r}_i^2 \right) \\
    U = \sum_i m_i \dot{r}_i^2 + C_2 \\
\end{aligned}
\end{aligned}\]

Where \(C_2\) is the total system energy. Plugging in our particle
system representation of \(U\), we find that the total system energy is
given by:

\[\begin{aligned}
\begin{aligned}
    C_2 &= \sum_i m_i \dot{r}_i^2 - \frac{1}{2} G \sum_i \sum_{j \neq i} \frac{m_i m_j}{r_{ji}} \\
\end{aligned}
\end{aligned}\]

\subsection{Fall 2023}\label{fall-2023}

Note that this problem was also given in Fall 2021.

\subsubsection{Problem Statement}\label{problem-statement-1}

Assuming Keplerian motion, several important types of orbital maneuvers
are noncoplanar. For example, the capability to change the ascending
node can widen the launch window.

Assume that the orbital elements for an Earth orbit are given.

\begin{enumerate}
\tightlist
\item
  To change only the ascending node, derive an equation (or equations)
  that, if solved, will identify the location. i.e.~the argument of
  latitude, for the location of the maneuver in the original and final
  orbits.
\item
  If the orbit is circular, let \(e=0.0\), \(i=55^\circ\),
  \(\Omega_i=0^\circ\), \(\Omega_f=45^\circ\), where \(o\) reflects the
  original orbit and \(f\) indicates a value in the final orbit. In the
  relationships from (a), demonstrate that the maneuver location is
  defined as \(\theta_o = 103.36^\circ\). What is the value of
  \(\theta_f\)?
\item
  If the circular orbit possesses a radius of \(3R_\oplus\), find the
  required \(\Delta v\).
\end{enumerate}

\subsubsection{Solution}\label{solution-1}

We can form a spherical triangle with side lengths
\(\Omega_f - \Omega_o\) along the equator, and then \(\theta_i\)
extending upwards from the left at an angle of \(i_o\), and \(\theta_f\)
extending upwards from the right at an angle of \(180^\circ-i_o\). The
angle at the top of the triangle is the angle between the initial and
final position vectors, which is the angle of the required \(\Delta v\).
We can then use the spherical law of cosines to solve for this angle:

\[\begin{aligned}
\begin{aligned}
    \cos a &= \cos b \cos c + \sin b \sin c \cos A \\
    \cos A &= - \cos b \cos c + \sin b \sin c \cos a \\
\end{aligned}
\end{aligned}\]

Where the lowercase letters are the side lengths and the uppercase
letters are the interior angles. Rephrased for our problem, we find the
third interior angle \(a_3\):

\[\begin{aligned}
\begin{aligned}
    \cos a_3 &= - \cos i_o \cos (180^\circ - i_f) + \sin i_o \sin (180^\circ - i_f) \cos(\Omega_f - \Omega_o) \\
    &= \cos^2 55^\circ + \sin^2 55^\circ \cos(45^\circ) \\
    &= \cos^2 55^\circ + \frac{\sqrt{2}}{2} \sin^2 55^\circ \\
    a_3 &= \cos^{-1} \left( \cos^2 55^\circ + \frac{\sqrt{2}}{2} \sin^2 55^\circ \right) \\
    &\approx 37^\circ \\
\end{aligned}
\end{aligned}\]

We can then use the spherical law of sines to solve for \(\theta_o\):

\[\begin{aligned}
\begin{aligned}
    \frac{\sin\theta_o}{\sin i_f} &= \frac{\sin(\Omega_f - \Omega_o)}{\sin a_3} \\
    \sin\theta_o &= \frac{\sin i_f \sin(\Omega_f - \Omega_o)}{\sin a_3} \\
    \theta_o &= 76.64^\circ \\
\end{aligned}
\end{aligned}\]

Notice that the arcsin is also solved by
\(\theta_o = 180^\circ - 76.64^\circ = 103.36^\circ\). We choose this
solution to yield an intersection in the first half of the initial
orbit.

Solving for \(\theta_f\) similarly:

\[\begin{aligned}
\begin{aligned}
    \frac{\sin\theta_f}{\sin i_o} &= \frac{\sin(\Omega_f - \Omega_o)}{\sin a_3} \\
    \sin\theta_f &= \frac{\sin i_o \sin(\Omega_f - \Omega_o)}{\sin a_3} \\
    \theta_f &= 76.64^\circ \\
\end{aligned}
\end{aligned}\]

We can then find the magnitude of the required \(\Delta v\) using the
law of cosines by recognizing that the magnitude of the velocity is the
same for both the initial and final orbits:

\[\begin{aligned}
\begin{aligned}
    v_c &= \sqrt{\frac{\mu_\oplus}{r}} \\
    &= \sqrt{\frac{\mu_\oplus}{3R_\oplus}} \\
\end{aligned}
\end{aligned}\]

And the magnitude of the required \(\Delta v\) is given by:

\[\begin{aligned}
\begin{aligned}
    \frac{\Delta v}{2 v_c} &= \sin\left( \frac{a_3}{2} \right) \\
    &= \sin\left( \frac{37^\circ}{2} \right) \\
    &\approx 0.30 \\
    \Delta v &\approx 0.60 v_c \\
    &\approx 0.60 \sqrt{\frac{\mu_\oplus}{3R_\oplus}} \\
    &\approx 2.86 \: [km/s] \\
\end{aligned}
\end{aligned}\]

This concludes the derivation of the ten integrals of motion for the
n-body problem. The first six scalars are the initial position and
velocity of the system, and the next three are the angular momentum of
the system. The final scalar is the total energy of the system.

\subsection{Problem 0}\label{problem-0}

\subsubsection{Problem Statement}\label{problem-statement-2}

In Keplerian mechanics, several important types of orbital maneuvers are
noncoplanar. For example, the capability to change both the ascending
node and the inclination with only one maneuver is efficient and can
widen the launch window.

Assume that the orbital elements for an Earth orbit are given. If the
orbit is circular both initially and after the maneuver, let
\(i_o=30^\circ\), \(i_f=90^\circ\), \(\Omega_o=0^\circ\),
\(\Omega_f=60^\circ\), where \(o\) reflects the original orbit and \(f\)
indicates a value in the final orbit.

\begin{enumerate}
\tightlist
\item
  Determine the appropriate maneuver location in each orbit.
\item
  If the circular orbit possesses a radius of \(4R_\oplus\), determine
  the magnitude of the required single impulse to accomplish the goal.
\end{enumerate}

\subsubsection{Solution}\label{solution-2}

We'll define the ``location'' of the maneuver in the initial and final
orbits with the argument of latitude \(\theta_o\) and \(\theta_f\), the
angle between the ascending node and the spacecraft's position vector.
Because the orbits are circular, we can't really use the true anomaly.
We can then form a spherical triangle with side lengths
\(\Omega_f - \Omega_o\) along the equator, and then \(\theta_i\)
extending upwards from the left at an angle of \(i_o\), and \(\theta_f\)
extending upwards from the right at an angle of \(i_f\). Note: that in
general, a spherical triangle has a sum of interior angles greater than
\(180^\circ\). This means that we must solve for the interior angle at
the top of the triangle using the spherical law of cosines:

\[\begin{aligned}
\begin{aligned}
    \cos a &= \cos b \cos c + \sin b \sin c \cos A \\
    \cos A &= - \cos b \cos c + \sin b \sin c \cos a \\
\end{aligned}
\end{aligned}\]

Where the lowercase letters are the side lengths and the uppercase
letters are the interior angles. Rephrased for our problem, we find the
third interior angle \(a_3\):

\[\begin{aligned}
\begin{aligned}
    \cos a_3 &= - \cos i_o \cos i_f + \sin i_o \sin i_f \cos(\Omega_f - \Omega_o) \\
    &= - \cos 30^\circ \cos 90^\circ + \sin 30^\circ \sin 90^\circ \cos(60^\circ - 0^\circ) \\
    &= - \frac{\sqrt{3}}{2} \cdot 0 + \frac{1}{2} \cdot 1 \cdot \frac{1}{2} \\
    &= \frac{1}{4} \\
    a_3 &= \cos^{-1} \left( \frac{1}{4} \right) \approx 76^\circ \\
\end{aligned}
\end{aligned}\]

Using the spherical law of sines, we can solve for \(\theta_o\):

\[\begin{aligned}
\begin{aligned}
    \frac{\sin\theta_o}{\sin i_f} &= \frac{\sin(\Omega_f - \Omega_o)}{\sin a_3} \\
    \sin\theta_o &= \frac{\sin i_f \sin(\Omega_f - \Omega_o)}{\sin a_3} \\
\end{aligned}
\end{aligned}\]

Plugging in values, we find:

\[\begin{aligned}
\begin{aligned}
    \sin\theta_o &= \frac{\sin 90^\circ \sin(60^\circ)}{\sin 76^\circ} \\
    &= \frac{\sin 60^\circ}{\sin 76^\circ} \\
    &\approx 0.89 \\
    \theta_o &\approx 63^\circ
\end{aligned}
\end{aligned}\]

And similarly for \(\theta_f\):

\[\begin{aligned}
\begin{aligned}
    \frac{\sin\theta_f}{\sin i_o} &= \frac{\sin(\Omega_f - \Omega_o)}{\sin a_3} = 1 \\
    \sin\theta_f &= \sin 30^\circ \frac{\sin(60^\circ)}{\sin 76^\circ} \\
    &\approx 0.63 \\
    \theta_f &\approx 26.5^\circ
\end{aligned}
\end{aligned}\]

The magnitude of the required impulse is given by the law of cosines,
where we know that the angle between the initial and final position
vectors is \(a_3 \approx 76^\circ\), the interior angle of the spherical
triangle at the point of intersection. The circular velocity in the
initial orbit is given by:

\[\begin{aligned}
\begin{aligned}
    v_c &= \sqrt{\frac{\mu_\oplus}{r}} \\
    &= \sqrt{\frac{\mu_\oplus}{4R_\oplus}} \\
\end{aligned}
\end{aligned}\]

And the magnitude of the required impulse is given by:

\[\begin{aligned}
\begin{aligned}
    \frac{\Delta v}{2 v_c} &= \sin\left( \frac{76^\circ}{2} \right) \\
    &\approx 0.62 \\
    \Delta v &\approx 1.23 v_c \\
    &\approx 1.23 \sqrt{\frac{\mu_\oplus}{4R_\oplus}} \\
\end{aligned}
\end{aligned}\]

\subsection{Fall 2019}\label{fall-2019}

\subsubsection{Problem Statement}\label{problem-statement-3}

Consider a hyperbolic flyby of a planet

\begin{enumerate}
\tightlist
\item
  Determine the values of the periapsis flyby radius \(r_p\) and
  hyperbolic excess speed \(v_\infty\) that yield the \emph{maximum
  possible} magnitude of the equivalent \(\Delta v_{eq}\) for the
  spacecraft due to the flyby. Express your answer for \(r_p\) in terms
  of the planet radius \(r_s\); include the constraint that
  \(r_p \geq r_s\).
\item
  Determine this maximum \(\Delta v_{eq}\) in terms of \(v_s\), the
  circular speed at the surface of the planet. Also determine the
  numerical values for the corresponding turn angle \(\delta\) and the
  hyperbolic eccentricity \(e\).
\end{enumerate}

\subsubsection{Solution}\label{solution-3}

We know that the angle between the incoming and outgoing hyperbolic
asymptotes is given by:

\[\begin{aligned}
\begin{aligned}
    \delta &= 2 \sin^{-1} \left( \frac{1}{e} \right) \\
    &= 2 \sin^{-1} \left( \frac{\Delta v_{eq}}{2 v_\infty} \right)
\end{aligned}
\end{aligned}\]

We'll use these two expressions for \(\delta\) to solve for the
conditions that maximize \(\Delta v\). First, we have to find a way to
introduce \(r_p\) into the equation. We know that the distance from the
attracting focus to the center of the hyperbola is given by:

\[\begin{aligned}
\begin{aligned}
    ae &= r_p + a \\
    e &= \frac{r_p}{a} + 1
\end{aligned}
\end{aligned}\]

We also know that by conservation of energy at \(r=\infty\), we can
express the semi-major axis \(a\) in terms of the hyperbolic excess
speed \(v_\infty\):

\[\begin{aligned}
\begin{aligned}
    \frac{v_\infty^2}{2} &= \frac{\mu}{2a} \\
    a &= \frac{\mu}{v_\infty^2}
\end{aligned}
\end{aligned}\]

Substituting this into the expression for \(e\):

\[\begin{aligned}
\begin{aligned}
    e &= \frac{r_p}{\mu/v_\infty^2} + 1 \\
    &= \frac{r_p v_\infty^2}{\mu} + 1
\end{aligned}
\end{aligned}\]

Such that we can equate the two expressions for \(\delta\):

\[\begin{aligned}
\begin{aligned}
    2 \sin^{-1} \left( \frac{\Delta v_{eq}}{2 v_\infty} \right) &= 2 \sin^{-1} \left( \frac{1}{\frac{r_p v_\infty^2}{\mu} + 1} \right) \\
    \frac{\Delta v_{eq}}{2 v_\infty} &= \frac{1}{\frac{r_p v_\infty^2}{\mu} + 1} \\
    \Delta v_{eq} &= \frac{2 v_\infty}{\frac{r_p v_\infty^2}{\mu} + 1} \\
\end{aligned}
\end{aligned}\]

This tells us that for any given \(v_\infty\), minimizing \(r_p\) will
maximize \(\Delta v_{eq}\). The minimum value of \(r_p\) is \(r_s\), the
radius of the planet. Solving for the \(v_\infty\) that corresponds to
this minimum \(r_p\) requires taking the derivative of the
\(\Delta v_{eq}\) expression with respect to \(v_\infty\) and looking
for critical points:

\[\begin{aligned}
\begin{aligned}
    \frac{\partial \Delta v_{eq}}{\partial v_\infty} &= \frac{2}{\frac{r_p v_\infty^2}{\mu} + 1} - \frac{2 v_\infty}{\left( \frac{r_p v_\infty^2}{\mu} + 1 \right)^2} \frac{2 r_p v_\infty}{\mu} \\
    &= \frac{\frac{2r_p v_\infty^2}{\mu} + 2 - \frac{4r_p v_\infty^2}{\mu}}{\left( \frac{r_p v_\infty^2}{\mu} + 1 \right)^2} \\
    &= \frac{2 - \frac{2r_p v_\infty^2}{\mu}}{\left( \frac{r_p v_\infty^2}{\mu} + 1 \right)^2} \\
\end{aligned}
\end{aligned}\]

We notice that the denominator is always positive, so we can simply set
the numerator to zero:

\[\begin{aligned}
\begin{aligned}
    2 - \frac{2r_p v_\infty^2}{\mu} &= 0 \\
    \frac{2r_p v_\infty^2}{\mu} &= 2 \\
    v_\infty^2 &= \frac{\mu}{r_p} \\
    v_\infty &= \sqrt{\frac{\mu}{r_p}}
\end{aligned}
\end{aligned}\]

This is an interesting result! We have found that the hyperbolic excess
velocity for maximum \(\Delta v_{eq}\) is equal to the circular velocity
at the surface of the planet. Solving for the corresponding value of
\(\Delta v_{eq}\):

\[\begin{aligned}
\begin{aligned}
    \Delta v_{eq} &= \frac{2 v_\infty}{\frac{r_p v_\infty^2}{\mu} + 1} \\
    &= \frac{2 \sqrt{\frac{\mu}{r_p}}}{\frac{r_p \left( \sqrt{\frac{\mu}{r_p}} \right)^2}{\mu} + 1} \\
    &= \frac{2 \sqrt{\frac{\mu}{r_p}}}{\frac{\mu}{\mu} + 1} \\
    &= \frac{2 \sqrt{\frac{\mu}{r_p}}}{2} \\
    &= \sqrt{\frac{\mu}{r_p}}
\end{aligned}
\end{aligned}\]

We can also solve for the corresponding values of \(\delta\):

\[\begin{aligned}
\begin{aligned}
    \delta &= 2 \sin^{-1} \left( \frac{1}{e} \right) \\
    &= 2 \sin^{-1} \left( \frac{\Delta v_{eq}}{2 v_\infty} \right) \\
    &= 2 \sin^{-1} \left( \frac{\sqrt{\frac{\mu}{r_p}}}{2 \sqrt{\frac{\mu}{r_p}}} \right) \\
    &= 2 \sin^{-1} \left( \frac{1}{2} \right) \\
    &= 60^\circ \\
\end{aligned}
\end{aligned}\]

And \(e\):

\[\begin{aligned}
\begin{aligned}
    e &= \frac{r_p}{a} + 1 \\
    &= \frac{r_p}{\frac{\mu}{v_\infty^2}} + 1 \\
    &= \frac{r_p v_\infty^2}{\mu} + 1 \\
    &= \frac{\mu}{\mu} + 1 \\
    &= 2
\end{aligned}
\end{aligned}\]

\section{Attitude Dynamics Past
Problems}\label{attitude-dynamics-past-problems}

\subsection{Fall 2023}\label{fall-2023-1}

This was the first problem written by Dr.~Oguri.

\subsubsection{Problem Statement}\label{problem-statement-4}

Let us analyically investigate the attitude motion of a satellite under
torque, \(\boldsymbol{L} \in \mathbb{R}^3\)

The inertial frame and satellite body-fixed frames are represented by
\(\mathcal{N}\) and \(\mathcal{B}\), where
\(\left\{\hat{\boldsymbol{n}}_1, \hat{\boldsymbol{n}}_2, \hat{\boldsymbol{n}}_3\right\}\)
and
\(\left\{\hat{\boldsymbol{b}}_1, \hat{\boldsymbol{b}}_2, \hat{\boldsymbol{b}}_3\right\}\)
are right-handed bases attached to the \(\mathcal{N}\) and
\(\mathcal{B}\) frames, respectively.

Denote the angular velocity of the \(\mathcal{B}\) frame with respect to
the \(\mathcal{N}\) frame as
\(\boldsymbol{\omega}_\mathcal{B/N} = \omega_i \hat{\boldsymbol{b}}_i\)
and the inertia tensor dyadic of the stellite about its center of mass
(CoM) by
\(\boldsymbol{I} = I_i \hat{\boldsymbol{b}}_i \hat{\boldsymbol{b}}_i\).

\begin{enumerate}
\item
  Derive the following equation from
  \(\dot{\boldsymbol{H}} = \boldsymbol{L}\), where \(\boldsymbol{H}\)
  represents the body's angular momentum about the CoM.

  \[\boldsymbol{I} \cdot \dot{\boldsymbol{\omega}}_\mathcal{B/N} = -\boldsymbol{\omega}_\mathcal{B/N} \times \boldsymbol{I} \cdot \boldsymbol{\omega}_\mathcal{B/N} + \boldsymbol{L}\]
\item
  Starting from the above equation, derive a set of differential
  equations that describe the time derivative of \(\omega_i\) for
  \(i = 1,2,3\), where \(\boldsymbol{L} = L_i \hat{\boldsymbol{b}}_i\).
\item
  Assume that \(I_1 < I_2 < I_3\) and \(\boldsymbol{L} = 0\). Show that
  a rotation about \(\hat{b}_1\) is a particular solution of the
  differential equations derived above, and discuss the linear stability
  of the solution; if it is linearly stable, also discuss whether or not
  the solution is asymptotically stable with the mathematical reasoning.
  Finally, qualitatively discuss whether the stability properties
  changes over time when there is energy dissipation (e.g., due to fuel
  sloshing).
\item
  Assume that \(I_1 > I_3 > I_2\) and now \(\boldsymbol{L}\) represents
  the gravity gradient (GG) torque in a circular orbit of radius \(R\)
  about a planet of gravitational parameter \(\mu\). ?Using an attitude
  representation of your choice, derive both the Kinematic and dynamic
  differential equations that govern the satellite motion. Here, an
  approximate epression of the GG torque,
  \(\boldsymbol{L} \approx \frac{3\mu}{R^5} \boldsymbol{R} \times \boldsymbol{I} \cdot \boldsymbol{R}\)
  where \(\boldsymbol{R} \in \mathbb{R}^3\) is the orbit radius vector,
  can be used without derivation. Show a particular solution of the
  derived differential equations and discuss its linear stability.
\end{enumerate}

\subsubsection{Solution}\label{solution-4}

Part 1:

Beginning with:

\[\dot{\boldsymbol{H}} = \boldsymbol{L}\]

We recognize that the angular momentum vector
\(\boldsymbol{H} = \boldsymbol{I} \boldsymbol{\omega}\) such that in the
body frame:

\[I \dot{\boldsymbol{\omega}} = \boldsymbol{L}\]

The BKE tells us that the time derivative of a vector in an reference
frame is related to the same vector's derivative in the rotating frame
by the angular velocity:

\[{}^\mathcal{N}\dot{\boldsymbol{v}} = {}^\mathcal{B}\dot{\boldsymbol{v}} + \boldsymbol{\omega}_\mathcal{B/N} \times \boldsymbol{v}\]

To be specific, when we say the derivative of a vector expressed in a
certain frame, like \({}^\mathcal{N}\dot{\boldsymbol{v}}\), we mean the
time derivative of the vectors elements when expressed in the basis
vectors of that frame. So, for example, if we have a vector
\(\boldsymbol{v} = v_i \hat{\boldsymbol{n}}_i\) expressed in the
\(\mathcal{N}\) frame, then
\({}^\mathcal{N}\dot{\boldsymbol{v}} = \dot{\boldsymbol{v}}_i \hat{\boldsymbol{n}}_i\)
as the basis vectors are constant in their frame. Applying this to the
angular momentum, we can take its body frame derivative:

\[\begin{aligned}
\begin{aligned}
    {}^\mathcal{N}\dot{\boldsymbol{H}} &= {}^\mathcal{B}\dot{\boldsymbol{H}} + \boldsymbol{\omega}_\mathcal{B/N} \times \boldsymbol{H} \\
    &= \boldsymbol{I} \dot{\boldsymbol{\omega}}_\mathcal{B/N} + \boldsymbol{\omega}_\mathcal{B/N} \times \boldsymbol{I} \boldsymbol{\omega}_\mathcal{B/N} \\
\end{aligned}
\end{aligned}\]

Since we know that
\({}^\mathcal{N}\dot{\boldsymbol{H}} = \boldsymbol{L}\), we can
substitute to yield:

\[\begin{aligned}
\begin{aligned}
    \boldsymbol{L} &= \boldsymbol{I} \dot{\boldsymbol{\omega}}_\mathcal{B/N} + \boldsymbol{\omega}_\mathcal{B/N} \times \boldsymbol{I} \boldsymbol{\omega}_\mathcal{B/N} \\
    \boldsymbol{I} \dot{\boldsymbol{\omega}}_\mathcal{B/N} &= -\boldsymbol{\omega}_\mathcal{B/N} \times \boldsymbol{I} \boldsymbol{\omega}_\mathcal{B/N} + \boldsymbol{L} \\
\end{aligned}
\end{aligned}\]

Completing the proof.

Part 2:

We can expand the above equation into its components, noting that if the
body basis vectors are the principal axes of the body, then the inertia
tensor is diagonal and the cross product terms are zero:

\[\begin{aligned}
\begin{bmatrix}
        I_1 & 0 & 0 \\
        0 & I_2 & 0 \\
        0 & 0 & I_3 \\
    \end{bmatrix}
    \begin{bmatrix}
        \dot{\boldsymbol{\omega}}_1 \\
        \dot{\boldsymbol{\omega}}_2 \\
        \dot{\boldsymbol{\omega}}_3 \\
    \end{bmatrix}
    =
    \begin{bmatrix}
        \boldsymbol{\omega}_1 \\
        \boldsymbol{\omega}_2 \\
        \boldsymbol{\omega}_3 \\
    \end{bmatrix}
    \times
    \begin{bmatrix}
        I_1 & 0 & 0 \\
        0 & I_2 & 0 \\
        0 & 0 & I_3 \\
    \end{bmatrix}
    \begin{bmatrix}
        \boldsymbol{\omega}_1 \\
        \boldsymbol{\omega}_2 \\
        \boldsymbol{\omega}_3 \\
    \end{bmatrix}
    +
    \begin{bmatrix}
        L_1 \\
        L_2 \\
        L_3 \\
    \end{bmatrix}
\end{aligned}\]

Such that the scalar EOMs are:

\[\begin{aligned}
\begin{aligned}
    \dot{\boldsymbol{\omega}}_1 &= \frac{1}{I_1} \left(\omega_2 \omega_3 \left(I_2 - I_3\right) + L_1 \right)\\
    \dot{\boldsymbol{\omega}}_2 &= \frac{1}{I_2} \left(\omega_3 \omega_1 \left(I_3 - I_1\right) + L_2 \right)\\
    \dot{\boldsymbol{\omega}}_3 &= \frac{1}{I_3} \left(\omega_1 \omega_2 \left(I_1 - I_2\right) + L_3 \right)\\
\end{aligned}
\end{aligned}\]

Part 3:

A rotation purely about \(\hat{\boldsymbol{b}}_1\) implies that
\(\boldsymbol{\omega} = \omega_1 \hat{\boldsymbol{b}}_1\) with
\(\omega_2=\omega_3 = 0\) such that the EOMs become when
\(\boldsymbol{L} = 0\):

\[\begin{aligned}
\begin{aligned}
    \dot{\boldsymbol{\omega}}_1 &= \frac{1}{I_1} \left(0 \cdot 0 \left(I_2 - I_3\right) + 0 \right)\\
    \dot{\boldsymbol{\omega}}_2 &= \frac{1}{I_2} \left(0 \cdot \omega_1 \left(I_3 - I_1\right) + 0 \right)\\
    \dot{\boldsymbol{\omega}}_3 &= \frac{1}{I_3} \left(\omega_1 \cdot 0 \left(I_1 - I_2\right) + 0 \right)\\
\end{aligned}
\end{aligned}\]

Which simplify:

\[\begin{aligned}
\begin{aligned}
    \dot{\boldsymbol{\omega}}_1 &= 0\\
    \dot{\boldsymbol{\omega}}_2 &= 0\\
    \dot{\boldsymbol{\omega}}_3 &= 0\\
\end{aligned}
\end{aligned}\]

This means that the angular velocity is constant in time, and therefore
this is a particular solution to the EOMs. To determine the stability of
this solution, we can linearize the EOMs about this solution by taking
the first order Taylor series expansion of the EOMs about the solution,
where we substitute
\(\boldsymbol{\omega} = \boldsymbol{\omega} + \delta\boldsymbol{\omega}\),
discarding any higher order terms in \(\delta\boldsymbol{\omega}\):

\[\begin{aligned}
\boldsymbol{\omega} = 
    \begin{bmatrix} \omega_1 + \delta\omega_1 \\ 0 + \delta\omega_2 \\ 0 + \delta\omega_3 \end{bmatrix}
\end{aligned}\]

\[\begin{aligned}
\begin{aligned}
    \dot{\boldsymbol{\omega}}_1 &= \frac{1}{I_1} \left(\left(\delta\omega_3 \cdot  \delta\omega_2\right) \left(I_2 - I_3\right) \right)\\
    \dot{\boldsymbol{\omega}}_2 &= \frac{1}{I_2} \left(\left(\left(\omega_1 + \delta\omega_1\right) \cdot \delta\omega_3\right) \left(I_3 - I_1\right) \right)\\
    \dot{\boldsymbol{\omega}}_3 &= \frac{1}{I_3} \left(\left(\left(\omega_1 + \delta\omega_1\right) \cdot \delta\omega_2\right) \left(I_1 - I_2\right) \right)\\
\end{aligned}
\end{aligned}\]

Simplifying:

\[\begin{aligned}
\begin{aligned}
    \dot{\boldsymbol{\omega}}_1 &= 0 \\
    \dot{\boldsymbol{\omega}}_2 &= \frac{1}{I_2} \left(\omega_1 \cdot \delta\omega_3 \left(I_3 - I_1\right) \right)\\
    \dot{\boldsymbol{\omega}}_3 &= \frac{1}{I_3} \left(\omega_1 \cdot \delta\omega_2 \left(I_1 - I_2\right) \right)\\
\end{aligned}
\end{aligned}\]

Finding the eigenvalues of this system will determine its linear
stability. We can rearrange this as a linear system in terms of the
perturbation \(\delta \omega_i\):

\[\begin{aligned}
\begin{aligned}
    \dot{\boldsymbol{\omega}} &= \boldsymbol{A} \delta \omega \\
    &= \begin{bmatrix}
        0 & 0 & 0 \\
        0 & 0 & \frac{1}{I_2} \left(\omega_1 \left(I_3 - I_1\right) \right)\\
        0 & \frac{1}{I_3} \left(\omega_1 \left(I_1 - I_2\right) \right) & 0 \\
    \end{bmatrix}
    \begin{bmatrix} \delta\omega_1 \\ \delta\omega_2 \\ \delta\omega_3 \end{bmatrix}
\end{aligned}
\end{aligned}\]

The eigenvalues of this linear system are given by the solution to:

\[\det\left(\boldsymbol{A} - \lambda \boldsymbol{I}\right) = 0\]

\[\begin{aligned}
\begin{aligned}
    \det\left(\begin{bmatrix}
        -\lambda & 0 & 0 \\
        0 & -\lambda & \frac{1}{I_2} \left(\omega_1 \left(I_3 - I_1\right) \right)\\
        0 & \frac{1}{I_3} \left(\omega_1 \left(I_1 - I_2\right) \right) & -\lambda \\
    \end{bmatrix}\right) &= 0 \\
    \lambda \left( \lambda^2 - \frac{\omega_1^2 \left(I_1 - I_2\right) \left(I_3 - I_1\right)}{I_2 I_3} \right) &= 0 \\
\end{aligned}
\end{aligned}\]

Which has solutions:

\[\begin{aligned}
\begin{aligned}
    \lambda_1 &= 0 \\
    \lambda_2 &= \sqrt{\frac{\omega_1^2 \left(I_1 - I_2\right) \left(I_3 - I_1\right)}{I_2 I_3}} \\
    \lambda_3 &= -\sqrt{\frac{\omega_1^2 \left(I_1 - I_2\right) \left(I_3 - I_1\right)}{I_2 I_3}} \\
\end{aligned}
\end{aligned}\]

Because we are told that \(I_1 < I_2 < I_3\), we know that
\(\left(I_1 - I_2\right) < 0\) and \(\left(I_3 - I_1\right) > 0\) such
that second and third eigenvalues are purely imaginary as the argument
under the square root must be negative. Due to the presence of a zero
eigenvalue, the system is marginally stable in the linear sense.

The system is not asymptotically stable as no eigenvalues have negative
real parts. This means that the system will not return to the
equilibrium solution after a perturbation. This makes intuitive sense as
torque-free rigid body motion has no damping capability to return the
system to the equilibrium solution.

Let's now turn out attention to the question of energy dissipation. We
know that the total kinetic energy of the system is given by:

\[T = \frac{1}{2} \boldsymbol{\omega} \cdot \boldsymbol{I} \cdot \boldsymbol{\omega}\]

We know that as long as no torque is applied to the system due to an
external force, the total angular momentum of the system is conserved.
Thinking back to the construction of the energy ellipsoid and momentum
sphere (expressed in body-frame angular momentum coordinates), losing
energy will shrink the energy ellipsoid nonlinearly along all its axes.
This could completely change the stability properties of the motion.
Shrinking the ellipsoid to the point where are one of its axes has the
same magnitude as the angular momentum sphere will create a directrix,
resulting in unstable motion about the intermediate axis.

Part 4:

The attitude representation of choice for this writeup is the direction
cosine matrix (DCM). We know that the DCM \(\left[\mathcal{BN}\right]\)
is defined as the matrix that takes vectors from the inertial frame to
the body frame:

\[{}^\mathcal{B}\boldsymbol{v} = \left[\mathcal{BN}\right] {}^\mathcal{N}\boldsymbol{v}\]

The kinematic differential equation for the DCM is a relationship
between the time derivative of the DCM and the angular velocity of the
body frame with respect to the inertial frame. We can derive a
relationship between these quantities by taking the
\(\mathcal{N}\)-frame derivative of the \(\mathcal{B}\)-frame basis
vectors:

\[\begin{aligned}
{}^\mathcal{N}\frac{d}{dt}\left(
        \begin{bmatrix}
            \hat{\boldsymbol{b}}_1 \\ \hat{\boldsymbol{b}}_2 \\ \hat{\boldsymbol{b}}_3
        \end{bmatrix}
    \right) = {}^\mathcal{B}\frac{d}{dt}\left(
        \begin{bmatrix}
            \hat{\boldsymbol{b}}_1 \\ \hat{\boldsymbol{b}}_2 \\ \hat{\boldsymbol{b}}_3
        \end{bmatrix}
    \right) + \boldsymbol{\omega}_\mathcal{B/N} \times \left(
        \begin{bmatrix}
            \hat{\boldsymbol{b}}_1 \\ \hat{\boldsymbol{b}}_2 \\ \hat{\boldsymbol{b}}_3
        \end{bmatrix}
    \right)
\end{aligned}\]

Here we note that the \(\mathcal{B}\)-frame derivative of the
\(\hat{\boldsymbol{b}}_i\) unit vectors is zero. Further, we can replace
the cross product on the right hand side with the matrix multiplication
of the skew symmetric matrix of the angular velocity with the basis
vectors:

\[\begin{aligned}
\begin{aligned}
    {}^\mathcal{N}\frac{d}{dt}\left(
        \begin{bmatrix}
            \hat{\boldsymbol{b}}_1 \\ \hat{\boldsymbol{b}}_2 \\ \hat{\boldsymbol{b}}_3
        \end{bmatrix}
    \right) &= [\boldsymbol{\omega}_\mathcal{B/N}\times]
        \begin{bmatrix}
            \hat{\boldsymbol{b}}_1 \\ \hat{\boldsymbol{b}}_2 \\ \hat{\boldsymbol{b}}_3
        \end{bmatrix} \\
        &= \begin{bmatrix}
            0 & -\omega_3 & \omega_2 \\
            \omega_3 & 0 & -\omega_1 \\
            -\omega_2 & \omega_1 & 0 \\
        \end{bmatrix}
        \begin{bmatrix}
            \hat{\boldsymbol{b}}_1 \\ \hat{\boldsymbol{b}}_2 \\ \hat{\boldsymbol{b}}_3
        \end{bmatrix} \\
\end{aligned}
\end{aligned}\]

We now proceed by computing the effect of the \(\mathcal{N}\)-frame
derivative on each of the \(\mathcal{B}\)-frame basis vectors, beginning
with \(\hat{\boldsymbol{b}}_1\):

\[\begin{aligned}
\begin{aligned}
    {}^\mathcal{N}\frac{d}{dt}\left(\hat{\boldsymbol{b}}_1\right) &= \begin{bmatrix}
            0 & -\omega_3 & \omega_2 \\
            \omega_3 & 0 & -\omega_1 \\
            -\omega_2 & \omega_1 & 0 \\
        \end{bmatrix}
        \begin{bmatrix}
            1 \\ 0 \\ 0
        \end{bmatrix} = \begin{bmatrix}
            0 \\ \omega_3 \\ -\omega_2
        \end{bmatrix} \\
\end{aligned}
\end{aligned}\]

\[\begin{aligned}
\begin{aligned}
    {}^\mathcal{N}\frac{d}{dt}\left(\hat{\boldsymbol{b}}_2\right) &= \begin{bmatrix}
            0 & -\omega_3 & \omega_2 \\
            \omega_3 & 0 & -\omega_1 \\
            -\omega_2 & \omega_1 & 0 \\
        \end{bmatrix}
        \begin{bmatrix}
            0 \\ 1 \\ 0
        \end{bmatrix} = \begin{bmatrix}
            -\omega_3 \\ 0 \\ \omega_1
        \end{bmatrix} \\
\end{aligned}
\end{aligned}\]

\[\begin{aligned}
\begin{aligned}
    {}^\mathcal{N}\frac{d}{dt}\left(\hat{\boldsymbol{b}}_3\right) &= \begin{bmatrix}
            0 & -\omega_3 & \omega_2 \\
            \omega_3 & 0 & -\omega_1 \\
            -\omega_2 & \omega_1 & 0 \\
        \end{bmatrix}
        \begin{bmatrix}
            0 \\ 0 \\ 1
        \end{bmatrix} = \begin{bmatrix}
            \omega_2 \\ -\omega_1 \\ 0
        \end{bmatrix} \\
\end{aligned}
\end{aligned}\]

At this point, we notice that since the DCM can be expressed as the dot
product of the basis vectors of the two frames:

\[\begin{aligned}
\left[\mathcal{BN}\right] = \begin{bmatrix}
        \hat{\boldsymbol{b}}_1 \cdot \hat{\boldsymbol{n}}_1 & \hat{\boldsymbol{b}}_1 \cdot \hat{\boldsymbol{n}}_2 & \hat{\boldsymbol{b}}_1 \cdot \hat{\boldsymbol{n}}_3 \\
        \hat{\boldsymbol{b}}_2 \cdot \hat{\boldsymbol{n}}_1 & \hat{\boldsymbol{b}}_2 \cdot \hat{\boldsymbol{n}}_2 & \hat{\boldsymbol{b}}_2 \cdot \hat{\boldsymbol{n}}_3 \\
        \hat{\boldsymbol{b}}_3 \cdot \hat{\boldsymbol{n}}_1 & \hat{\boldsymbol{b}}_3 \cdot \hat{\boldsymbol{n}}_2 & \hat{\boldsymbol{b}}_3 \cdot \hat{\boldsymbol{n}}_3 \\
    \end{bmatrix}
\end{aligned}\]

Which is really just rows of \(\mathcal{B}\)-frame vectors expressed in
the \(\mathcal{N}\)-frame:

\[\begin{aligned}
\left[\mathcal{BN}\right] = \begin{bmatrix}
        {}^\mathcal{N}\hat{\boldsymbol{b}}_1 \\ {}^\mathcal{N}\hat{\boldsymbol{b}}_2 \\ {}^\mathcal{N}\hat{\boldsymbol{b}}_3 \\
    \end{bmatrix}
\end{aligned}\]

Differentiating:

\[\begin{aligned}
\begin{aligned}
    \left[\dot{\mathcal{BN}}\right] = \begin{bmatrix}
        {}^\mathcal{N}\frac{d}{dt}\left(\hat{\boldsymbol{b}}_1\right) \\ {}^\mathcal{N}\frac{d}{dt}\left(\hat{\boldsymbol{b}}_2\right) \\ {}^\mathcal{N}\frac{d}{dt}\left(\hat{\boldsymbol{b}}_3\right) \\
    \end{bmatrix} = \begin{bmatrix}
        0 & \omega_3 & -\omega_2 \\
        -\omega_3 & 0 & \omega_1 \\
        \omega_2 & -\omega_1 & 0 \\
    \end{bmatrix}
    \begin{bmatrix}
        {}^\mathcal{N}\hat{\boldsymbol{b}}_1 \\ {}^\mathcal{N}\hat{\boldsymbol{b}}_2 \\ {}^\mathcal{N}\hat{\boldsymbol{b}}_3 \\
    \end{bmatrix} \\
\end{aligned}
\end{aligned}\]

Such that:

\[\dot{\left[\mathcal{BN}\right]} = -\left[\boldsymbol{\omega}_\mathcal{B/N}\times\right] \left[\mathcal{BN}\right]\]

Deriving the dynamic differential equation in terms of the DCM requires
us to express the orbital radius vector (defined in the orbital frame as
\(\boldsymbol{R} = R \hat{o}_1\)):

\[\begin{aligned}
\begin{aligned}
    {}^\mathcal{B}\boldsymbol{R} &= \left[\mathcal{BO}\right] \begin{bmatrix} R \\ 0 \\ 0 \end{bmatrix} \\
    &= R \begin{bmatrix} C_{11} \\ C_{21} \\ C_{31} \end{bmatrix} \\
\end{aligned}
\end{aligned}\]

Such that the gravity gradient torque is:

\[\begin{aligned}
\begin{aligned}
    \boldsymbol{L} &= \frac{3\mu}{R^5} \boldsymbol{R} \times \boldsymbol{I} \cdot \boldsymbol{R} \\
    &= \frac{3\mu}{R^3} \begin{bmatrix} C_{11} \\ C_{21} \\ C_{31} \end{bmatrix} \times \begin{bmatrix} I_1 & 0 & 0 \\ 0 & I_2 & 0 \\ 0 & 0 & I_3 \end{bmatrix} \begin{bmatrix} C_{11} \\ C_{21} \\ C_{31} \end{bmatrix} \\
    &= \frac{3\mu}{R^3} \begin{bmatrix} C_{21} C_{31} \left(I_3 - I_2\right) \\ C_{31} C_{11} \left(I_1 - I_3\right) \\ C_{11} C_{21} \left(I_2 - I_1\right) \end{bmatrix} \\
\end{aligned}
\end{aligned}\]

Plugging this into the EOMs:

\[\begin{aligned}
\begin{aligned}
    \boldsymbol{I} \cdot \dot{\boldsymbol{\omega}}_\mathcal{B/N} &= -\boldsymbol{\omega}_\mathcal{B/N} \times \boldsymbol{I} \cdot \boldsymbol{\omega}_\mathcal{B/N} + \boldsymbol{L} \\
    \begin{bmatrix} I_1 & 0 & 0 \\ 0 & I_2 & 0 \\ 0 & 0 & I_3 \end{bmatrix} \begin{bmatrix} \dot{\boldsymbol{\omega}}_1 \\ \dot{\boldsymbol{\omega}}_2 \\ \dot{\boldsymbol{\omega}}_3 \end{bmatrix} &= \begin{bmatrix} \omega_2 \omega_3 \left(I_2 - I_3\right) \\ \omega_3 \omega_1 \left(I_3 - I_1\right) \\ \omega_1 \omega_2 \left(I_1 - I_2\right) \end{bmatrix} + \frac{3\mu}{R^3} \begin{bmatrix} C_{21} C_{31} \left(I_3 - I_2\right) \\ C_{31} C_{11} \left(I_1 - I_3\right) \\ C_{11} C_{21} \left(I_2 - I_1\right) \end{bmatrix} \\
\end{aligned}
\end{aligned}\]

A particular solution of these equations occurs when the body is
oriented such that one of its principal axes is in line with the orbital
radius vector and body is rotating at the same rate as the orbit
(\(\Omega\)). If we choose to align the largest moment of inertia body
axis (\(I_1\)) with the orbital radius vector, then we can simplify the
above equations by noticing that \(C_{11} = 1\),
\(C_{21} = C_{31} = 0\), while \(\omega_3 = \Omega\),
\(\omega_2 = \omega_1 = 0\):

\[\begin{aligned}
\begin{aligned}
    \begin{bmatrix} I_1 & 0 & 0 \\ 0 & I_2 & 0 \\ 0 & 0 & I_3 \end{bmatrix} \begin{bmatrix} \dot{\boldsymbol{\omega}}_1 \\ \dot{\boldsymbol{\omega}}_2 \\ \dot{\boldsymbol{\omega}}_3 \end{bmatrix} &= \begin{bmatrix} 0 \\ 0 \\ 0 \end{bmatrix} 
\end{aligned}
\end{aligned}\]

This particular solution is expected to be unstable as any small
rotation about \(\hat{\boldsymbol{b}}_3\) will create an additional
torque that will cause large-scale oscillations. This is illustrated by
drawing the scenario and showing that any small reorientation will
produce a net torque which is not restoring.

\section{Orbit Determination Past
Problem}\label{orbit-determination-past-problem}

\end{document}
